\documentclass{report}

% Language setting
\usepackage[italian]{babel}

% Set page size and margins
% Replace `letterpaper' with `a4paper' for UK/EU standard size
\usepackage[letterpaper,top=2cm,bottom=2cm,left=3cm,right=3cm,marginparwidth=1.75cm]{geometry}

% Useful packages
\usepackage[numbers]{natbib}
\usepackage{amsmath}
\usepackage{graphicx}
\usepackage[colorlinks=true, allcolors=blue]{hyperref}

\usepackage{tcolorbox}

\begin{document}

\begin{titlepage}
    \centering
    \includegraphics[width=0.25\textwidth]{logoPoliba.png} 
    
    \vspace{1cm}
    
    % --- INTESTAZIONE ISTITUZIONE ---
    {\large \textbf{POLITECNICO DI BARI}} \\
    \vspace{0.5cm}
    {\small DIPARTIMENTO DI INGEGNERIA ELETTRICA E DELL'INFORMAZIONE} \\
    {\small Corso di Laurea in Ingegneria Informatica e dell'Automazione} \\
    
    \vspace{0.4cm}
    
    % Linea orizzontale superiore
    \hrule 
    
    \vspace{1.5cm}
    
    % --- TIPO DI TESI ---
    {\large Progetto di ingegneria del Software} \\
    
    \vspace{2.5cm}
    
    % --- TITOLO DELLA TESI ---
    {\Huge \textbf{Progettazione di una mappa, basata su Community, per supporto a Persone con Mobilita Ridotta}} \\
    % "Persona a Mobilità Ridotta" (P.M.R.) È il termine ufficiale utilizzato a livello internazionale (in inglese PRM - Persons with Reduced Mobility) nei trasporti, negli aeroporti e nella progettazione urbana.
    % Perché usarlo: Include non solo chi è in sedia a rotelle, ma anche chi ha difficoltà temporanee, usa stampelle, deambulatori o è anziano con passo lento. È perfetto per un'app di navigazione.

    \vspace{4cm}
    
    % --- RELATORE E LAUREANDO ---
    % Uso due minipage per allinearli a sinistra e destra
    \noindent
    \hfill
    \begin{minipage}[t]{0.45\textwidth}
        \begin{flushright}
            \large Team:\\
            \textbf{Dario Esposito} \\
            \textbf{Mauro Lavigna} \\
            \textbf{Alessandro Pio Casto} \\
        \end{flushright}
    \end{minipage}
    
    \vfill % Spinge tutto il resto verso il fondo pagina
    
    % --- PIÈ DI PAGINA ---
    \hrule
    \vspace{0.5cm}
    Anno Accademico 2025 - 2026
    
\end{titlepage}

%\maketitle

\begin{abstract}
Negli ultimi anni, la salute degli anziani è in miglioramento, come riportano i report ISTAT, ma la disabilità e l’invecchiamento della popolazione hanno reso sempre più importante sviluppare strumenti digitali che favoriscano la mobilità, l'autonomia e l’inclusione sociale. Sebbene esistano numerose applicazioni per la navigazione, esse non tengono sempre conto delle esigenze specifiche degli utenti anziani, con mobilità ridotta o con disabilità.
Il progetto proposto consiste nella realizzazione di un'applicazione accessibile e intuitiva che integri un sistema di geolocalizzazione, pensato per aiutare gli anziani e le persone con disabilità a individuare luoghi facilmente raggiungibili, percorsi accessibili e servizi specifici sul territorio.
Le attuali applicazioni di navigazione presentano aspetti generici in cui le persone a mobilità ridotta presentano difficoltà nell’utilizzo, in quanto c’è uno scarso supporto alle interazioni e una fruizione poco accessibile.
\end{abstract}

\tableofcontents

%\chapter*{Introduzione}

\chapter{Contestualizzazione di un problema spesso dimenticato}

\section{Alcuni dati di interesse}

Non c'è alcun dubbio che lo spostamento per persone a Mobilità Ridotta sia esponenzialmente
migliorato negli ultimi decenni, anche a fronte di una esigenza pratica. Come riportato dai seguenti dati ISTAT\cite{istat_1},
il 15.8\% degli adulti di età compresa tra 65 e 74 anni presentano delle limitazioni motorie; la percentuale
aumenta proporzionalmente con la fascia di età. Un altro aspetto di cui è necessario discutere, in quanto verrà trattato successivamente
con dettaglio, è la questione dell'alfabetizzazione digitale relativa a persone appartenenti a fasce di età avanzate. I seguenti dati ISTAT\cite{istat_2}
riportano come le competenze digitali di base per cittadini italiani di età compresa tra i 55 e 64 anni siano superiori di quelli appartenenti alla
fascia 65-74.

\begin{figure}[h]
    \centering
    \includegraphics[width=0.25\textwidth]{Materiale Multimediale/Foto/imad-alassiry unsplash-Feb-5-2022.jpg}%
    \hspace{0.05\textwidth}
    \includegraphics[width=0.35\textwidth]{Materiale Multimediale/Foto/max-bender-unsplash-Maggio-11-2020.jpg}
    \caption{Anziana con deambulatore pieghevole \cite{foto_deambulatore_mobile} e adulto con mobility scooter \cite{foto_mobility_scooter}}
    \label{fig:due_immagini}
\end{figure}

\section{La scarsa mobilità come problema non univoco}

Quando si tratta il problema della mobilità finalizzato a persone a mobilità ridotta, la soluzione auspicata sarebbe quella che
risolverebbe ciascuna esigenza. Le esigenze saranno differenti a seconda della situazione in cui il praticolare individuo si trova.
In generale, per i nostri scopi, considereremo le seguenti tre categorie di impedimenti:

\begin{enumerate}
    \item Impedimenti visivi.
    \item Impedimenti motori.
    \item Impedimenti uditivi.
\end{enumerate}

Le tre tipologie di impedimento, nella pratica, fanno sorgere differenti necessità; ad esempio persone affette da impedimenti
visivi rilevano difficoltà nell'accedere informazioni in loco, come ad esempio in una stazione ferroviaria, oppure individui con impedimenti
motori avranno difficoltà ad accedere fisicamente a luoghi di interesse, con potenziali disagi legati a strumentazione malfunzionante
(ad esempio ascensori)\cite{art_1}. Chiarire preventivamente le necessità pratiche che ciascuna fascia richiede ha un duplice ruolo; in primis
rende la soluzione applicabile a differenti situazioni ed in secondo luogo si attenuerebbe il disagio per persone affette da diversi impedimenti.

\chapter{Stato dell'arte}

E' indubbio che la tecnologia e l'avanzamento di quest'ultima abbia agevolato molti aspetti della vita di un individuo,
indipendentemente dalla situazione di salute di quest'ultimo. Precedentemente si era accennato alla alfabetizzazione digitale relativa a persone
appartenenti a fasce d'età avanzate, molto spesso le stesse affette dagli impedimenti precedentemente elencati, in particolare quelli motori; in Francia, si stima che 
il numero di persone mobilitanti con sedia a rotelle tende ad aumentare esponenzialemente in funzione dell'età \cite{art_2}.
Con l'avanzare del tempo si auspica che la percentuale di individui appartenenti a tali fasce che disponga di competenze digitali di base continui ad aumentare. 
Questo è dovuto al fatto che molte soluzioni tendono ad essere, in modo letterale, alla portata di mano, utilizzabili da dispositivi palmari e, negli ultimi anni,
anche in smartwatch. Tuttavia nella quotidianità, è opportuno citare alcune soluzioni valide anche in assenza di dispositivi mobili.
Uno dei più noti esempi di dispositivo per persone a mobilità ridotta, in particolare rivolto a persone affette da impedimenti visivi, sono collocati
in prossimità di alcuni attraversamenti pedonali. Questi ultimi costituiscono una fonte di pericolo, soprattutto per una persona ipovedente in quanto possono non essere
adeguatamente in grado di gestire il passaggio. Negli ultimi anni in molte città italiane iniziano ad essere presenti dei dispositivi che notificano acusticamente
la possibilità (o l'impossibilità) di attraversamento.


\section{OpenStreetMap}

Negli ultimi anni, come precedentemente accennato, le tecnologie permettono di minimizzare la problematica
direttamente con l'uso di dispositivi mobili, come ad esempio cellulari. La tecnologia di base a cui si farà riferimento è nota come \textbf{OpenStreetMap}, che
permette l'accesso a mappe ed informazioni geografiche. La peculiarità di OpenStreetMap è che, oltre ad essere completamente open source, costituisce una soluzione
a cui chiunque può contribuire. Una possibile obiezione relativa a questo aspetto è che le informazioni a cui è possibile accedere possono non essere completamente affidabili
o peggio del tutto errate. Questo costituirebbe un enorme rischio, soprattutto se il fine è quello di gestire la mobilità per persone affette da impedimenti. Per cui riteniamo doveroso
dover analizzare in qualche modo l'affidabilità di questa tecnologia. Si rileva che le informazioni confinate su OpenStreetMap risultano essere particolarmente accurate; in Inghilterra una posizione
su OpenStreetMap risulta essere compresa in un raggio di 6 metri dalla rispettiva posizione reale.\cite{art_3}. E' rilevante menzionare che la qualità delle informazioni geografiche
presenti su OpenStreetMap non è la medesima per ogni luogo; il dataset relativo alla Germania risulta essere quello di più alta qualità \cite{art_4}, ciò dipende dal numero di utenti
che contribuiscono attivamente; per quanto riguarda la situazione in Italia, per ogni milione di persone il numero di utenti che contribuiscono sono compresi tra 1.63 a 3.35.
In alternativa, esistono tecnologie non open source che garantiscono dati geografici più accurati, come ad esempio Google Maps Platform; in seguito verrà effettuata una stima dei costi
nel caso in cui il progetto proposto dovesse implementare Google Maps Platform.

\section{Wheelmap}

Molte sono le applicazioni che si basato su OpenStreetMap, una di queste è Wheelmap, finalizzata ad ottimizzare la mobilità per persone affette da impedimenti motori. L'applicazione permette di mostrare
luoghi generici che possono essere più o meno accessibili a persone dotate di supporti per la mobilità. Wheelmap offre livelli di granularità relativi all'accessibilità dei luoghi, infatti si possono visualizzare
luoghi che sono parzialmente accessibili, parzialmente accessibili con la presenza di servizi igienici accessibili, totalmente accessibili, totalmente accessibili con servizi igienici
accessibili e luoghi che risultano non accessibili.

\begin{figure}[h]
    \centering
    \includegraphics[width=0.65\textwidth]{Materiale Multimediale/Foto/Interfaccia_wheelmap.png}%
    \hspace{0.05\textwidth}
    \caption{Interfaccia di Wheelmap \cite{wheelmap_interfaccia}.}
    \label{fig:immagini}
\end{figure}

Similmente a OpenStreetMap, anche Wheelmap permette agli utenti di contribuire attivamente alla piattaforma, ad esempio alla selezione di un luogo di interesse è possibile per un utente aggiungere fotografie del
dato luogo. Sempre alla selezione di un luogo, è possibile inserire una valutazione, positiva o negativa, ai servizi igienici del dato luogo; prima di inserire il giudizio viene ricordato all'utente
i vari parametri che stabiliscono se un servizio igienico è accessibile in sedia a rotelle o meno.

\section{Screen Reader}

Per screen reader si intende un software che permette di convertire il testo, o più generalemente il contenuto, visualizzato su uno schermo in un output audio. Questo risulta essere particolarmente utile per persone affette da impedimenti visivi che non
hanno la possibilità di leggere in autonomia il contenuto del dispositivo. Tuttavia, questa tecnologia presenta alcune limitazioni; le informazioni che vengono erogate sono all'incirca il 61.48\% 
meno accurate e si spende il 210.96\% in più per interagirvi rispetto a utenti che non utilizzano screen reader\cite{art_5}. Molti degli attuali sistemi operativi per dispositivi mobili presentano degli screen reader
integrati, come ad esempio VoiceOver per dispositivi Apple e TalkBack per dispositivi Android.

\section{BrailleTouch}

BrailleTouch costituisce un esempio di applicazione per dispositivi mobili atta a facilitare la navigazione per utenti con impedimenti visivi\cite{art_6}. L'applicazione si basa sul
noto linguaggio Braille, linguaggio tattile che permette ad una persona ipovedente di leggere e scrivere interpretando dei punti di rilievo opportunamente distribuiti in una matrice avente 3 righe e 2 colonne. Un aspetto peculiare di
BrailleTouch è che gli stessi sviluppatori la pongono come una soluzione intermedia tra due tipologie di soluzioni, quelle hardware, costituite da maggiori performance ma anche da costi elevati, e soluzioni software, che presentano minori performance ma costi più accessibili.
Si è ritenuto opportuno analizzare il caso di BrailleTouch in quanto, nonostante l'applicazione non abbia come finalità diretta quella di facilitare la mobilità, aiuta a comprendere i modi in cui le persone affette da impedimenti visivi preferiscono interagire con dispositivi mobili,
e ciò potrebbe rappresentare uno spunto per possibili sviluppi futuri.

\begin{figure}[h]
    \centering
    \includegraphics[width=0.35\textwidth]{Materiale Multimediale/Foto/BrailleTouchs-input-surface.jpg}%
    \hspace{0.05\textwidth}
    \caption{Schermata di input di BrailleTouch\cite{art_6}.}
    \label{fig:immaginii}
\end{figure}

\section{Kimap}

Kimap è un'applicazione per dispositivi mobili che, per certi versi, più si avvicina all'applicazione che intendiamo realizzare. E' utile menzionare il fatto che Kimap prevede
una versione per browser, tuttavia con limitazioni rispetto alla versione mobile. Un'aspetto cardine dell'applicazione è il concetto della community, infatti è possibile
iscriversi ed unirsi alla community per avere una serie di vantaggi, tra cui la possibilità di salvare luoghi preferiti, partecipare a crowdmapping (si intende l'aggregazione di dati geografici da parte di utenti differenti)
e raccogliere dei punti. 

\begin{figure}[h]
    \centering
    \includegraphics[width=0.27\textwidth]{Materiale Multimediale/Foto/Interfaccia Kimap 1.jpeg}%
    \includegraphics[width=0.27\textwidth]{Materiale Multimediale/Foto/Interfaccia Kimap 2.jpeg}%
    \hspace{0.05\textwidth}
    \caption{Interfacce di Kimap.}
    \label{fig:immaginii}
\end{figure}

Su quest'ultimo punto è necessario soffermarsi, infatti Kimap implementa elementi di \textbf{Gamification}, quale utilizzo può aumentare considerevolmente il coinvolgimento degli utenti\cite{art_7}, anche per applicazioni
che esulano dal mondo della didattica e dell'apprendimento. Come è possibile osservare dalle immagini seguenti, i punti verranno assegnati all'utente qualora quest'ultimo contribuisca in qualche modo alla piattaforma, si distinguono:

\begin{enumerate}
    \item \textit{Lasciare una recesione}: +10 punti.
    \item \textit{Lasciare una recensione con in allegato una foto del posto}: +25 punti.
    \item \textit{Recensire un posto in funzione della sua accessibilità}: +20 punti.
\end{enumerate}

Qualora un determinato account dovesse raggiungere quota 2000 punti, l'utente diventerà un "Kimapper esperto" e con esso si potranno ottenere specifici distintivi.

\begin{figure}[h]
    \centering
    \includegraphics[width=1\textwidth]{Materiale Multimediale/Foto/Foto_kimap_menzione_gamification.png}%
    \hspace{0.05\textwidth}
    \caption{Sito web di Kimap che menziona l'aspetto di Gamification implementato.}
    \label{fig:immaginii}
\end{figure}


\chapter{Organizzazione della soluzione proposta}

Data la natura del problema, il modello di processo che si è preferito implementare è lo Scrum; questa scelta nasce da diverse esigenze tra cui il doversi adattare
a potenziali nuove tecnologie da dover implementare senza che questo rechi disagi nello sviluppo generale del progetto. Un'altra motivazione cardine è legata alla documentazione richiesta;
a differenza di altri approcci, il team ha gestito la stesura della documentazione in maniera contemporanea allo sviluppo del codice. Questo si è rivelato un enorme vantaggio in termini di tempo.
Lo Scrum giornaliero veniva effettuato considerando due ore accademiche (complessivamente 1 ora e 40 minuti)

\section{Valutazioni generali e rischi}

\begin{figure}[h]
    \centering
    \includegraphics[width=0.95\textwidth]{Materiale Multimediale/Gantt_foto.png}%
    \hspace{0.05\textwidth}
    \caption{Diagramma di Gantt utilizzato.}
    \label{fig:immaginii}
\end{figure}

Uno dei possibili rischi è correlato alla variabilità dei requisiti. Questo è legato al fatto
che, nel caso in cui il team fosse venuto a conoscenza di una funzionalità che avrebbe permesso il miglioramento dell'applicazione, si sarebbe
optato per la sua implementazione. Un altro fattore di rischio è legato a soluzioni già esistenti; per certi versi Wheelmap può
costituirne un esempio concreto, tuttavia l'applicazione proposta si differenzia da Wheelmap in quanto permetterà di creare un profilo su cui
ciascun utente potrà gestire al meglio la navigazione; l'esperienza personalizzata risulta una caratteristica più che adeguata 
soprattutto considerando le differenti esigenze di cui ciascun utente necessiterà.

\section{Stima dei costi}

Un'aspetto fondamentale è quantificare quanto capitale sia necessario investire affinchè l'applicazione sia funzionante e disponibile. Una precisazione doverosa riguarda i cosiddetti
acquisti in-app; al momento della stesura del documento corrente, non sono previsti acquisti effettuabili all'interno dell'applicazione. In secondo luogo si è ritenuto, per motivi 
precedentemente chiarificati, valutare anche l'utilizzo di un dataset geografico più accurato quale Google Maps Platform. L'aspetto cardine a cui faremo riferimento sarà dunque la disponibilità dell'applicazione,
in particolare quanto sarà necessario investire per mantenere l'applicazione accessibile presso i vari store di applicazione.

\subsection{Gli store di applicazioni}

Gli store di applicazioni rappresentano una piattaforma digitale in cui vengono resi disponibili applicativi, dove questi utlimi possono avere un determinato costo oppure possono essere
gratuiti.
Gli store di applicazioni presenti nei vari dispositivi mobili, costituiscono una parte fondamentale della soluzione proposta in quanto, a nostro avviso, il migliore utilizzo dell'applicazione 
si avrà proprio quando l'utente si interfaccerà con dispositivi che potrà utilizzare dal vivo. Per cui, risulta necessario introdurre i principali store di applicazioni su cui la soluzione proposta
sarà resa disponibile.

\begin{enumerate}
    \item \textit{Apple App Store}: è la piattaforma ufficiale dei dispositivi Apple.
    \item \textit{Google Play Store}: è la piattaforma ufficiale per dispositivi aventi sistema operativo Android.
\end{enumerate}

\begin{figure}[h]
    \centering
    \includegraphics[width=0.8\textwidth]{Materiale Multimediale/Foto/statistica_app_store_google_play_store.png}%
    \hspace{0.05\textwidth}
    \caption{Numero, in miliardi, di applicazioni scaricate nei diversi store di applicazioni\cite{statistica_store}.}
    \label{fig:immaginiii}
\end{figure}

Come si può notare dalla figura precedente, il numero di applicazioni scaricate sia nel Google Play Store che nell'App Store è considerevole.
Per questo si è ritenuto sufficiente considerare soltanto questi due store di applicazioni. In entrambi i casi, per poter pubblicare un'app si necessiterà
di un account sviluppatore, e ciò comporterà dei costi di cui è necessario effettuare una stima. Relativamente all'Apple App Store, la distribuzione potrà
avvenire a fronte di un costo annuale di 99 dollari. In maniera differente, il Google Play Store prevede un unico costo di 25 dollari.


\subsection{Stima dei costi utilizzando OpenStreetMap}

Come chiarificato in precedenza, l'utilizzo di OpenStreetMap non implica la presenza di costi aggiuntivi relativi ai dati geografici. Per mantenere disponibile
l'applicazione nell'Apple App Store e nel Google Play Store sarebbero necessari 124 dollari nel primo anno, per poi essere di 99 dollari dal secondo anno a segursi.

\subsection{Stima dei costi utilizzando Google Maps Platform}

Nel caso in cui l'applicazione implementi il dataset geografico più accurato di Google Maps Platform, il costo di mantenimento dell'applicazione risulta essere più elevato.
Sono previsti tre diversi piani tariffari:

\begin{enumerate}
    \item \textit{Starter:} 100 dollari al mese.
    \item \textit{Essentials:} 275 dollari al mese.
    \item \textit{Pro:} 1200 dollari al mese.
\end{enumerate}

E' utile menzionare il fatto che Google Maps Platform offre un credito pari a 300 dollari per i nuovi utenti che desiderano utilizzare il servizio;
questa opzione risulta particolarmente interessante in quanto permetterebbe di verificare e valutare di prima mano se l'utilizzo di Google Maps Platform
ricopre le esigenze del progetto proposto.


\section{Architettura}
Il progetto è stato sviluppato utilizzando un'architettura \textbf{client-server}, che consente una separazione evidente tra la gestione dell'interfaccia utente e la logica applicativa.
Il lato Client è rappresentato dall'applicazione mobile sviluppata in Kotlin, attraverso la quale l'utente può effettuare la registrazione e il login.
I dati inseriti dall'utente, vengono inviati al server, che ne gestisce la memorizzazione all'interno di un database.
Il server si occupa della gestione del database, il quale è stato integrato all'interno di un ambiente Docker per garantire maggiore portabilità.
Inoltre il server espone un'API, che rappresenta l'interfaccia di comunicazione tra client e backend e consente, a seguito del corretto completamento del processo di login, l'accesso alle funzionalità dell'applicazione,
inclusa la visualizzazione della mappa fornnita tramite un service esterno.

\subsection{Stile Architetturale lato client}

Vista la necessità di avere un sistema reattivo e interattivo, l'applicazione dovrebbe adottare un pattern \textbf{MVC}, che isola in tre livelli il sistema.
Il "Model" di occupa di gestire ed elaborare i dati, il "View" si occupa della modalità di visualizzazione dei dati mentre il "Controller" si occupa dell'interazione dell'utente e manda queste informazioni a vista e modello.
\newline
Dato che questo tipo di pattern potrebbe richiedere, specialmente nello sviluppo di applicativi mobili, grossi file che fungono sia da view che da controller,
si potrebbe optare per un pattern di tipo \textbf{MVVM} (Model-View-ViewModel). In questo tipo di modello abbiamo il ruolo del ViewModel che non conosce la View, semplicemente espone le informazioni, 
sarà poi il View a leggere ed interpretare queste informazioni.
\newline
La differenza sostanziale è che nel MVC il Controller dice direttamente cosa fare alla view, ma nel MVVM il ruolo del controller è sostituito dal ViewModel, che semplicemente rende disponibili le informazioni alla View, che gestirà in autonomia l'interfaccia.

\subsubsection{UI Layer (View)}

Si occupa di renderizzare la grafica, è interamente realizzato con il Framework dichiarativo Jetpack Compose in liguaggio \textbf{kotlin}.
È presente un'unica attività \texttt{LandingActivity.kt} che gestice le varie schermate, suddivise in schermate di Autenticazione (\texttt{LandingScreen.kt RegisterScreen.kt LoginScreen.kt}) e rendering delle mappe (\texttt{MapView.kt}).
\newline
L'oggetto che si occupa di navigare tra le varie schermate è il \texttt{NavController}, esso però non viene passato tra le varie schermate, ma ogni schermata evoca un evento che viene poi gestito dalla schermata principale.
Ad essere passato tra le varie schermate è il ViewModel, infatti per ogni schermata, nella di una generica cassella di testo abbiamo:


\begin{tcolorbox}[colback=gray!20, colframe=gray!0, title=Schermata di autenticazione]
\begin{verbatim}
OutlinedTextField(
            value = viewModel.password,
            onValueChange = {viewModel.password= it },
...
\end{verbatim}
\end{tcolorbox}

Quindi l'informazione viene salvata nell'oggeto ViewModel e poi la schermata si occupa di recuperare l'informazione

\subsubsection{Presentation Layer (ViewModel)}

Il modulo \texttt{UtenteViewModel.kt} si occupa di salvare le informazioni di autenticazione come email e password che vengono poi viste dal View.
\newline
Un altro ruolo importante di questo modulo è determinare in che stato si trova attualmente l'app con quattro stati definiti da una classe \textbf{sealed} (sigillata) ovvero con un numero specifico di figli, nè piu nè meno. Simile all'utilizzo di un enum ma con il vantaggio di poter includere altri attributi, diversi per ogni figlio.

\begin{tcolorbox}[colback=gray!20, colframe=gray!0, title=Schermata di autenticazione]
\begin{verbatim}
sealed class UiState {
    object Idle : UiState()                         // Nessuna operazione
    object Loading : UiState()                      // Caricamento
    data class Success(val msg: String) : UiState() // Operazione riuscita
    data class Error(val msg: String) : UiState()   // Errore
}
\end{verbatim}
\end{tcolorbox}

Il ViewModel ha il compito di eseguire il lavoro di rete in maniera asincrona, senza bloccare l'interfaccia.
Inolte ha anche la possibilità di verificare i dati immessi prima ancora di fare una chiamata di rete.

%da verificare ed integrare

\subsubsection{Model}
Sempre nel modulo \texttt{UtenteViewModel.kt} possiamo individuare la classe \texttt{NetworkClient} che viene creata con la parola chiave \texttt{object} proprio perché è di tipo \textbf{Singleton}, ovvero la creazione di una sola istanza di questa classe nell'intera app.
\newline
Nella stessa classe è stata usata la parola chiave  \texttt{by lazy} (\textbf{inizzializzazione pigra}) poiché questo comando permette di creare l'istanza di Retrofit non all'avvio dell'app ma quando si verifica la prima chiamata di rete

\section{Teck Stack}
\subsection{Ambiente di sviluppo}
L'IDE utilizzato per lo sviluppo è \textbf{Android Studio}, il quale fornisce strumenti per la scrittura del codice, consente il debugging, 
gestisce il testing delle applicazioni e le dipendenze tramite Grandle, ossia uno strumento di build utilizzato per automatizzare il processo di compilazione del progetto e nel nostro caso lo utilizziamo per gestire l'implementazione di librerie esterne presenti nel progetto.
Inoltre  la particolarità di android è l'integrazioni di  emulatori e strumenti di analisi che facilitano lo sviluppo e la manutenzione del software anche dal punto vista grafico e visivo, visualizzando in maniera instantanea l'applicativo progettato.
\subsection{Linguaggio di Programmazione}
Il linguaggio utilizzato per la realizzazione della web app è \textbf{Kotlin}.E' un linguaggio moderno e supportato per le applicazioni android.
E' progetatto per essere interoperabile con Java e supporta la programmazione sia orientata agli oggetti sia a paradigmi funzionali. Inoltre è un 
liguaggio staticamente tipizzato ed utilizza un vasto insieme di librerie e framework per velocizzare lo sviluppo di applicazioni
\subsection{Framework e librerie}
Per lo sviluppo dell'applicazione è stato utilizzato \textbf{Android Jetpack} al fine di garantire un'architettura scalabile e manutenibile.
L'interfaccia grafica è stata realizzata interamente con \textbf{Jetpack Compose}, il quale permette di descrivere la UI come funzione dello 
stato corrente,ciò lo fa essere più semplice da utilizzare. Per la gestione dei dati legati alla UI è stato utilizzato \textbf{ViewModel} , 
il quale garantisce che i dati, nonostante possano subire variazioni, non vengono persi.
La navigazione tra le diverse schermate dell'applicativo è gestita tramite \textbf{Navigation Compose}, ossia una libreria che permette di 
controllare il passaggio dei parametri tra le schermate in modo coerente con il pattern di navigazione di Android.Per l'integrazione con il server backend, 
è stata utilizzata la libreria \textbf{Retrofit}, il quale permette di trasformare le API HTTP in interfaccia Kotlin, automatizzando la gestione delle richieste e delle risposte, 
inoltre permette la conversione automatica dei dati JSON provenienti dal database in oggetti scritti in lunguaggio Kotlin
\subsection{API Service}
Per la visualizzazine della mappa sull'applicazione è stata utilizzata la libreria di \textbf{MapLibre}, un potente motore di rendering per visualizzare mappe vettoriali interattive.
MapLibre nasce da Mapbox GL, divenuto closed a partire dal 2020.
\newline
Alla base di MapLibre c'è uno stile, ovvero un file JSON definisce quali informazini mostrare e come disegnarle.
Lo stile puo essere modificato a proprio piacimento o può essere fornito da terzi, gratuitamente o a pagamento.
Per questa applicazione il fornitore è open-source ed è \textbf{OpenFreeMap} \footnote{\url{https://tiles.openfreemap.org/styles/bright}}, che recupera le informazini da OpenStreetMap.

Purtroppo OpenFreeMap non fornisce tutte le informazini necessarie al nostro fine.
Proprietà come \textcolor{green}{\texttt{"wheelchair"}} o \textcolor{green}{\texttt{"toilet:wheelchair"}}, importanti per verificare l'accessibilità di un luogo, non sono fornite, invece attingere direttamente alla Main API di OpenStreetMap risulterebbe troppo lento, a causa della sua natura open, infatti esso permette operazioni CRUD, perfette per la modifica ma con un'infrastruttura poco performante che non supporta query complesse.
\newline
Un database di sola lettura, ottimizzato alla ricerca, è \textbf{Overpass API} \footnote[2]{\url{https://wiki.openstreetmap.org/wiki/Overpass_API}}.
Con un'interrogazione in linguaggio Overpass QL è possibile ottenere una risposta in JSON.
È possibile testare le proprie query sul portale \url{https://overpass-turbo.eu/}.

Se \textbf{MapLibre} è il motore grafico, attraverso \textbf{OpenFreeMap} e \textbf{OverPass API}, le informazioni da visualizzare sono fornite da \textbf{OpenFreeMap}

\subsection{Tool di sviluppo}
Per la gestione del controllo versione è stato utilizzato \textbf{Git}, un software utilizzato localmente per gestire le modifiche apportate al progetto. Git supporta operazioni di branching e merging e permette,
 attraverso il commit, di tracciare ogni modifica del file in modo tale da rendere più efficace e rapida l'individuazione di eventuali errori critici.
Per la gestione remota della repository del progetto è stato utilizzato \textbf{GitHub}, ossia una piattaforma di hosting che consente di condividere il codice online, facilita il tutto attraverso le operazioni di pull request. 
Inoltre legge e visualizza  un file chiamato ".gitignore" il quale permette a git di evitare il caricamento di file temporanei generati automaticamente da Android Studio.
Per la gestione del server è stato utilizzato \textbf{Docker},una tecnologia basata sulla containerizzazione che permette di creare immagini che vengono eseguite in un ambiente isolato così facendo il backend non interferisce con il sistema operativo dell'host.
Attraverso il file "docker-compose.yml", è stato possibile avviare con un unico comando l'intero stack includendo il database relazionale e il web server per la gestione delle chiamate api.L'uso di Docker inoltre assicura la portabilità e semplifica le operazioni di rollback grazie al versionamento delle immagini.
All'interno di questa infrastruttura è stato configurato l'ambiente \textbf{XAMPP}, che funge da middleware per la gestione dei dati. Il database relazione si occupa della memorizzazione delle informazioni,
mentre la comunicazione tra l'applicazione android e il database è garantita dal serevr Apache, il quale riceve le richieste dalla libreria Retrofit. Queste richieste vengono elaborate in linguaggio PHP e convertono i risultati in formato JSON.

\section{Prototipo}
L'idea progettuale iniziale prevedeva la realizzazione di un'applicazione in grado di offrire all'utente sia la possibilità di registrarsi e creare un account personale,con successivo accesso tramite autenticazione, sia la possibilità di utilizzare l'applicazione senza la necessità di registrarsi. Una volta effettuato l'accesso, l'utente avrebbe dovuto visualizzare una mappa fornita da un'API esterna, all'interno della quale erano rappresentati diversi luoghi.
Per ciascun luogo era prevista l'indicazione del livello di accessibilità per persone con mobilità ridotta, oltre alla presenza di recensioni associate e al calcolo del percorso ottimale per raggiungerlo. Il progetto nella sua versione finale mantiene la possibilità di accesso sia tramite registrazione e login sia senza autenticazione. a seguito dell'accesso all'applicazione, viene visualizzata una mappa contenente i luoghi di interesse.
Rispetto all'idea progettuale iniziale, la versione finale dell'applicazione introduce un elemento migliorativo non precedentemente previsto, rappresentato dall'utilizzo di indicatori grafici di colore differente per identificare i luoghi sulla mappa in base al loro livello di accessibilità. Tale soluzione consente una comprensione immediata e intuitiva delle informazioni, migliorando l'esperienza utente e l'usabilità complessiva dell'applicazione. L'impiego di una codifica cromatica(verde,giallo,rosso e grigio) costituisce pertanto un'implementazione aggiuntiva rispetto alla concezione progettuale un'implementazione aggiuntiva rispetto alla concezione progettuale originaria, arricchendo la rappresentazione visiva dei dati senza alterare le funzionalità di base previste. In particolare, il colore verde indica luoghi completamente accessibili alle persone con mobilità ridotta, il giallo segnala una parziale accessibilità, il rosso identifica luoghi non accessibili, mentre il grigio indica l'assenza di informazioni relative all'accessibilità.
\newline
Rispetto all'idea progettuale iniziale, nella versione finale non sono state implementate le funzionalità relative alla gestione delle recensioni né il calcolo del percorso ottimale per raggiungere i luoghi selezionati.

\begin{figure}[h]
    \centering
    \includegraphics[width=0.25\textwidth]{schermataInii.jpeg}%
    \hspace{0.05\textwidth}
    \caption{Schermata iniziale}
    \label{fig:immaginii}
\end{figure}
\begin{figure}[h]
    \centering
    \includegraphics[width=0.25\textwidth]{login.jpeg}%
    \hspace{0.05\textwidth}
    \caption{Schermata per effettuare il login}
    \label{fig:immaginii}
\end{figure}
\begin{figure}[h]
    \centering
    \includegraphics[width=0.25\textwidth]{registrazione.jpeg}%
    \hspace{0.05\textwidth}
    \caption{Schermata per poter effettuare la registrazione}
    \label{fig:immaginii}
\end{figure}
\begin{figure}[h]
    \centering
    \includegraphics[width=0.25\textwidth]{tutti.jpeg}%
    \hspace{0.05\textwidth}
    \caption{Schermata visualizzata una volta aver effettuato l'accesso all'applicazione}
    \label{fig:immaginii}
\end{figure}
\begin{figure}[h]
    \centering
    \includegraphics[width=0.25\textwidth]{verde.jpeg}%
    \hspace{0.05\textwidth}
    \caption{Informazione in merito ad un luogo scelto nel caso in cui sia accessibile a tutti}
    \label{fig:immaginii}
\end{figure}
\begin{figure}[h]
    \centering
    \includegraphics[width=0.25\textwidth]{rosso.jpeg}%
    \hspace{0.05\textwidth}
    \caption{Informazione in merito ad un luogo scelto nel caso in cui non sia accessibile a tutti}
    \label{fig:immaginii}
\end{figure}


%Inizio bibliografia

\bibliographystyle{plainnat}

\bibliography{bibliografia}

\end{document}
