\documentclass{report}

% Language setting
\usepackage[italian]{babel}

% Set page size and margins
% Replace `letterpaper' with `a4paper' for UK/EU standard size
\usepackage[letterpaper,top=2cm,bottom=2cm,left=3cm,right=3cm,marginparwidth=1.75cm]{geometry}

% Useful packages
\usepackage[numbers]{natbib}
\usepackage{amsmath}
\usepackage{graphicx}
\usepackage[colorlinks=true, allcolors=blue]{hyperref}

\begin{document}

\begin{titlepage}
    \centering
    \includegraphics[width=0.25\textwidth]{logoPoliba.png} 
    
    \vspace{1cm}
    
    % --- INTESTAZIONE ISTITUZIONE ---
    {\large \textbf{POLITECNICO DI BARI}} \\
    \vspace{0.5cm}
    {\small DIPARTIMENTO DI INGEGNERIA ELETTRICA E DELL'INFORMAZIONE} \\
    {\small Corso di Laurea in Ingegneria Informatica e dell'Automazione} \\
    
    \vspace{0.4cm}
    
    % Linea orizzontale superiore
    \hrule 
    
    \vspace{1.5cm}
    
    % --- TIPO DI TESI ---
    {\large Progetto di ingegneria del Software} \\
    
    \vspace{2.5cm}
    
    % --- TITOLO DELLA TESI ---
    {\Huge \textbf{Progettazione di una mappa, basata su Community, per supporto a Persone con Mobilita Ridotta}} \\
    % "Persona a Mobilità Ridotta" (P.M.R.) È il termine ufficiale utilizzato a livello internazionale (in inglese PRM - Persons with Reduced Mobility) nei trasporti, negli aeroporti e nella progettazione urbana.
    % Perché usarlo: Include non solo chi è in sedia a rotelle, ma anche chi ha difficoltà temporanee, usa stampelle, deambulatori o è anziano con passo lento. È perfetto per un'app di navigazione.

    \vspace{4cm}
    
    % --- RELATORE E LAUREANDO ---
    % Uso due minipage per allinearli a sinistra e destra
    \noindent
    \hfill
    \begin{minipage}[t]{0.45\textwidth}
        \begin{flushright}
            \large Team:\\
            \textbf{Dario Esposito} \\
            \textbf{Mauro Lavigna} \\
            \textbf{Alessandro Pio Casto} \\
        \end{flushright}
    \end{minipage}
    
    \vfill % Spinge tutto il resto verso il fondo pagina
    
    % --- PIÈ DI PAGINA ---
    \hrule
    \vspace{0.5cm}
    Anno Accademico 2025 - 2026
    
\end{titlepage}

%\maketitle

\begin{abstract}
Negli ultimi anni, la salute degli anziani è in miglioramento, come riportano i report ISTAT [1], ma la disabilità e l’invecchiamento della popolazione hanno reso sempre più importante sviluppare strumenti digitali che favoriscano la mobilità, l'autonomia e l’inclusione sociale. Sebbene esistano numerose applicazioni per la navigazione, esse non tengono sempre conto delle esigenze specifiche degli utenti anziani, con mobilità ridotta o con disabilità.
Il progetto proposto consiste nella realizzazione di un'applicazione accessibile e intuitiva che integri un sistema di geolocalizzazione, pensato per aiutare gli anziani e le persone con disabilità a individuare luoghi facilmente raggiungibili, percorsi accessibili e servizi specifici sul territorio.
Le attuali applicazioni di navigazione presentano aspetti generici in cui le persone a mobilità ridotta presentano difficoltà nell’utilizzo, in quanto c’è uno scarso supporto alle interazioni e una fruizione poco accessibile.
\end{abstract}

\tableofcontents

\chapter*{Introduzione}

\chapter{Contestualizzazione di un problema spesso dimenticato}

\section{Alcuni dati di interesse}

Non c'è alcun dubbio che lo spostamento per persone a Mobilità Ridotta sia esponenzialmente
migliorato negli ultimi decenni, anche a fronte di una esigenza pratica. Come riportato dai seguenti dati ISTAT\footnote{https://www.istat.it/it/files/2021/07/Report-anziani-2019.pdf},
il 15.8\% degli adulti di età compresa tra 65 e 74 anni presentano delle limitazioni motorie; la percentuale
aumenta proporzionalmente con la fascia di età. Un altro aspetto di cui è necessario discutere, in quanto verrà trattato successivamente
con dettaglio, è la questione dell'alfabetizzazione digitale relativa a persone appartenenti a fasce di età avanzate. I seguenti dati ISTAT\footnote{https://www.istat.it/it/files/2023/06/cs-competenzedigitali.pdf}
riportano come le competenze digitali di base per cittadini italiani di età compresa tra i 55 e 64 anni siano superiori di quelli appartenenti alla
fascia 65-74.

\begin{figure}[h]
    \centering
    \includegraphics[width=0.25\textwidth]{Materiale Multimediale/Foto/imad-alassiry unsplash-Feb-5-2022.jpg}%
    \hspace{0.05\textwidth}%
    \includegraphics[width=0.35\textwidth]{Materiale Multimediale/Foto/max-bender-unsplash-Maggio-11-2020.jpg}
    \caption{Anziana con deambulatore pieghevole \cite{foto_deambulatore_mobile} e adulto con mobility scooter \cite{foto_mobility_scooter}}
    \label{fig:due_immagini}
\end{figure}

\section{Tecnologie adottate nel tempo}


Negli ultimi anni la mobilità è esponenzialmente migliorata. Ciò è dovuto ad una serie di motivazioni, tra cui una più frequente manutenzione da parte 
delle istituzioni dei vari segnali stradali. Ad esempio, la segnaletica orizzontale può essere considerata un ottimo modo per minimizzare
i sinistri stradali, specialmente se associata a proprietà retroriflettenti.



%Inizio bibliografia

\bibliographystyle{plainnat}

\bibliography{bibliografia}

\end{document}